\documentclass[tikz, border=10pt]{standalone}
\usepackage[utf8]{inputenc} 
\usepackage[ngerman]{babel}

\usepackage{tikz}
\usepackage{tkz-euclide}

% schöne Pfeilspitzen
\tikzset{%
	>=latex,
	inner sep=0pt,%
	outer sep=0pt,%         Das ist auch der Abstand zwischen Messgeräten und Leitungen!!!
	mark coordinate/.style={inner sep=0pt,outer sep=0pt,minimum size=3pt,
		fill=black,circle}%
}

\usepackage{siunitx}				            % A comprehensive (SI) units package
	\sisetup{output-decimal-marker = {,}}	% Komma als Dezimaltrennzeichen für siunitx
	\sisetup{exponent-product = \cdot}		% Punkt statt x zwischen Zahl und Zehnerpotenz
	\sisetup{per-mode = fraction, fraction-function=\tfrac}			% Bruchstrich in Einheit
	\sisetup{inter-unit-product = \cdot}
	\sisetup{detect-weight=true, detect-family=true} 
	\sisetup{detect-all=true}             % für fettes SI - Quelle: https://tex.stackexchange.com/questions/610211/how-to-have-bold-unit-with-siunitx

\begin{document}
\begin{tikzpicture}%[show background grid]
%	\useasboundingbox (-5,-1) rectangle (5,5);
%	\tkzClipBB
    
\tkzDefPoint(-2,0){S} 	\tkzDrawPoints[color=blue](S)	% \tkzLabelPoint[below left=1ex, color=blue](S){S} 
\tkzDefPoint(2,1){A}	\tkzDrawPoints[color=blue](A)	% \tkzLabelPoint[below left=1ex, color=blue](A){A}

% 	Berechnen der Streckenlänge SA
\tkzCalcLength(S,A){r}  \tkzGetLength{r}

\tkzDrawLines[add = 0.3 and 0.3, color=blue!100](S,A)

\tkzDrawArc[R,arc](S,1*\r)(5,90)	
\tkzDrawArc[R,arc](A,1*\r)(120,205)	

\tkzInterCC[R](S,\r)(A,\r)  \tkzGetFirstPoint{B_1}\tkzGetSecondPoint{B_2}
\tkzDrawPoints[color=blue](B_1)  

\tkzDrawLines[add = 0.3 and 0.3, color=blue!100](S,B_1)
\tkzMarkAngle[->, color=blue, size=1.5, draw,  thick](A,S,B_1)
\tkzFindAngle(A,S,B_1)      \tkzGetAngle{winkelalpha}
\tkzLabelAngle[color=blue](A,S,B_1){$ \pgfmathprintnumber{\winkelalpha}\si{\degree}$}
    
\tkzDefLine[bisector](A,S,B_1) \tkzGetPoint{a}%	\tkzDrawPoints[color=blue](a)	\tkzLabelPoints[below left=1ex, color=blue](a) 
% 	Anzeigen der Konstruktionslinien für die Winkelhalbierende
\tkzShowLine[bisector,size=3,gap=5, thick, color=gray](A,S,B_1)
% 	Winkelhalbierende einzeichnen
\tkzDrawLines[add=0 and 0.5,color=red!100](S,a)

\tkzFindAngle(A,S,a)        \tkzGetAngle{alphahalbe}
\tkzMarkAngle[->, color=magenta, size=2.6, draw,  thick](A,S,a)
\tkzLabelAngle[color=magenta, pos=2](A,S,a){$ \pgfmathprintnumber{\alphahalbe}\si{\degree}$}
            
\tkzFindAngle(a,S,B_1)      \tkzGetAngle{alphahalbe}
\tkzMarkAngle[->, color=magenta, size=2.4, draw,  thick](a,S,B_1)
\tkzLabelAngle[color=magenta, pos=2](a,S,B_1){$ \pgfmathprintnumber{\alphahalbe}\si{\degree}$}
\end{tikzpicture}
\end{document}

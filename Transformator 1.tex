\documentclass[border=10pt]{standalone}
\usepackage[utf8]{inputenc} 
\usepackage[ngerman]{babel}

\usepackage{tikz}
\usepackage{tkz-euclide}
\usetikzlibrary{3d, angles, arrows, arrows.meta, babel, backgrounds, bending, calc, circuits.logic.US, decorations.markings, decorations.pathmorphing, decorations.pathreplacing, decorations.text, decorations, er, fit, graphs, intersections, matrix, mindmap, patterns, plotmarks, positioning, quotes, svg.path, shadows.blur, shapes.misc, tikzmark, trees, shapes, shadows, through, circuits.logic.US, circuits.logic.IEC, circuits.logic.CDH, circuits.ee.IEC}
\usepackage[american, europeanresistors,americanvoltages,emptydiodes,siunitx]{circuitikz}
	\tikzset{circuit declare symbol = ammeter}
	\tikzset{set ammeter graphic ={draw,generic circle IEC, minimum size=5.5mm,info=center:A}}
	\tikzset{circuit declare symbol = voltmeter}
	\tikzset{set voltmeter graphic ={draw,generic circle IEC, minimum size=5.5mm,info=center:V}}

%   TikZ Grid Paper With Node Coordinates
\makeatletter% from https://tex.stackexchange.com/a/39698/121799
\def\grd@save@target#1{%
	\def\grd@target{#1}}
\def\grd@save@start#1{%
	\def\grd@start{#1}}
\tikzset{
	labeled grid/.style={
		to path={%
			\pgfextra{%
				\edef\grd@@target{(\tikztotarget)}%
				\tikz@scan@one@point\grd@save@target\grd@@target\relax
				\edef\grd@@start{(\tikztostart)}%
				\tikz@scan@one@point\grd@save@start\grd@@start\relax
				\draw[minor help lines] (\tikztostart) grid (\tikztotarget);
				\draw[major help lines] (\tikztostart) grid (\tikztotarget);
				\grd@start
				\pgfmathsetmacro{\grd@xa}{\the\pgf@x/1cm}
				\pgfmathsetmacro{\grd@ya}{\the\pgf@y/1cm}
				\grd@target
				\pgfmathsetmacro{\grd@xb}{\the\pgf@x/1cm}
				\pgfmathsetmacro{\grd@yb}{\the\pgf@y/1cm}
				\pgfmathsetmacro{\grd@xc}{\grd@xa + \pgfkeysvalueof{/tikz/grid with coordinates/major step}}
				\pgfmathsetmacro{\grd@yc}{\grd@ya + \pgfkeysvalueof{/tikz/grid with coordinates/major step}}
				%\foreach \x in {\grd@xa,\grd@xc,...,\grd@xb}
				%\node[anchor=north] at (\x,\grd@ya) {\pgfmathprintnumber{\x}};
				%\foreach \y in {\grd@ya,\grd@yc,...,\grd@yb}
				%\node[anchor=east] at (\grd@xa,\y) {\pgfmathprintnumber{\y}};
				\path foreach \x in {\grd@xa,\grd@xc,...,\grd@xb}
				{foreach \y in {\grd@ya,\grd@yc,...,\grd@yb}
					{ (\x,\y) node[grid with coordinates/grid label] {$(\pgfmathprintnumber{\x},\pgfmathprintnumber{\y})$}}};
			}
		}
	},
	minor help lines/.style={
		help lines,
		step=\pgfkeysvalueof{/tikz/grid with coordinates/minor step},
		draw=none
	},
	major help lines/.style={
		help lines,
		line width=\pgfkeysvalueof{/tikz/grid with coordinates/major line width},
		step=\pgfkeysvalueof{/tikz/grid with coordinates/major step},
		draw=none
	},
	grid with coordinates/.cd,
	minor step/.initial=.2,
	major step/.initial=1,
	major line width/.initial=0pt,
	grid label/.style={below,scale=0.3,color = black!20}
}
\makeatother


 
\begin{document}
%   Transformator 1
\begin{circuitikz}[circuit ee IEC, font=\sffamily\footnotesize, scale=1.0]

% Einblenden von Punktkoordinaten - ggf. auskommentieren
\draw (-1,-1) to[labeled grid] (8,4);

%   Primärspule
\draw 
(0,1.75) 	
to [short,o-]														(0,3)	
to [ammeter]														(2,3)	
to [short,-]														(3,3)
to [american inductor,color=green!50!black]							(3,0)
to [short,-]														(0,0)
to [short,-o]                                                       (0,1.25)	
;
\node at (1,3.6) {$I_\text{P}$}
;	
\draw 
(2,0) 
to [voltmeter] 														(2,3)
;
\node at (1.4,1.5) {$U_\text{P}$}
;	
\draw[fill=black] (2,0) circle (1.5pt)
;
\draw[fill=black] (2,3) circle (1.5pt)
;
\draw 
(5,0) 
to [voltmeter] 														(5,3)
;
\node at (5.6,1.5) {$U_\text{S}$}
;	
\draw[fill=black] (5,0) circle (1.5pt)
;
\draw[fill=black] (5,3) circle (1.5pt)
;

%   Sekundärspule
\draw 
(7,0) 	
to [short,o-]														(4,0)	
to [american inductor,color=red]                                    (4,3)
to [short]                                                          (5,3)	
to [ammeter]                                                        (7,3)
to [short,-o]														(7,3) 
;
\node at (6,3.6) {$I_\text{S}$}
;	

%   Eisenkern
\draw[line width = 1pt, color = blue] 
    (3.45,0.85) 	
    to [short, -]																	(3.45,2.15)	
    ;
\draw[line width = 1pt, color = blue] 
    (3.55,0.85) 	
    to [short, -]																	(3.55,2.15)	
    ;
\node at (0,1.5) {$\sim$}
;	
\node at (2.75,2) {$n_\text{P}$}
;	
\node at (4.25,2) {$n_\text{S}$}
;	
\end{circuitikz}
\end{document}

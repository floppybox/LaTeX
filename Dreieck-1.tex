\documentclass[tikz, border=10pt]{standalone}
\usepackage[utf8]{inputenc} 
\usepackage[ngerman]{babel}

\usepackage{tikz}
\usetikzlibrary{backgrounds}
\usepackage{tkz-euclide}

\usepackage{pgfplots}
\pgfplotsset{compat=newest}

\usepackage{siunitx}				        			% A comprehensive (SI) units package
	\sisetup{output-decimal-marker = {,}}					% Komma als Dezimaltrennzeichen für siunitx
	\sisetup{exponent-product = \cdot}					% Punkt statt x zwischen Zahl und Zehnerpotenz
	\sisetup{per-mode = fraction, fraction-function=\tfrac}			% Bruchstrich in Einheit
	\sisetup{inter-unit-product = \cdot}
	\sisetup{detect-weight=true, detect-family=true} 
	\sisetup{detect-all=true}             					% für fettes SI - Quelle: https://tex.stackexchange.com/questions/610211/how-to-have-bold-unit-with-siunitx

\begin{document}

\tikzstyle{background grid}=[draw, black!20,step=5mm,line width=1pt]
\begin{tikzpicture}[scale=1.0, show background grid]
\pgfkeys{/pgf/number format/.cd, fixed zerofill, precision=1, use comma, 1000 sep={.}}

\def\xA{0}
\def\yA{0.5}
\def\xB{4}
\def\yB{1}
\def\xC{2}
\def\yC{5}

\coordinate (A) at (\xA,\yA);
\coordinate (B) at (\xB,\yB);
\coordinate (C) at (\xC,\yC);

% Berechnung der Winkel aus den Punktkoordinaten
\pgfmathsetmacro \winkelalpha{{atan(abs((((\yB-\yA)/(\xB-\xA) - (\yC-\yA)/(\xC-\xA)) / (1 + (\yC-\yA)/(\xC-\xA) * (\yB-\yA)/(\xB-\xA)))))}}
\pgfmathsetmacro \winkelbeta{{atan(abs((((\yC-\yB)/(\xC-\xB) - (\yA-\yB)/(\xA-\xB)) / (1 + (\yC-\yB)/(\xC-\xB) * (\yA-\yB)/(\xA-\xB)))))}}
\pgfmathsetmacro \winkelgamma{{atan(abs((((\yA-\yC)/(\xA-\xC) - (\yB-\yC)/(\xB-\xC)) / (1 + (\yA-\yC)/(\xA-\xC) * (\yB-\yC)/(\xB-\xC)))))}}

\draw node[anchor=east, xshift=0mm, yshift=0mm] at (A)  {A};
\draw node[anchor=west, xshift=0mm, yshift=0mm] at (B)  {B};
\draw node[anchor=south, xshift=0mm, yshift=0mm] at (C)  {C};
	
\draw[color=black, line cap=round, line width=1pt] (A) -- (B);	
\draw[color=black, line cap=round, line width=1pt] (B) -- (C);	
\draw[color=black, line cap=round, line width=1pt] (C) -- (A);	

\tkzMarkAngle[->, color=blue, size=1.5, draw, thick](B,A,C)
\tkzMarkAngle[->, color=blue, size=1.5, draw, thick](C,B,A)
\tkzMarkAngle[->, color=blue, size=1.5, draw, thick](A,C,B)

\tkzLabelAngle[color=black, dist=1.0](B,A,C){$ \pgfmathprintnumber{\winkelalpha}\si{\degree}$}
\tkzLabelAngle[color=black, dist=0.9](C,B,A){$ \pgfmathprintnumber{\winkelbeta}\si{\degree}$}
\tkzLabelAngle[color=black, dist=1.1](A,C,B){$ \pgfmathprintnumber{\winkelgamma}\si{\degree}$}
 
\end{tikzpicture}

\end{document}

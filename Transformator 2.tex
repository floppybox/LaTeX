\documentclass[border=10pt]{standalone}
\usepackage[utf8]{inputenc} 
\usepackage[ngerman]{babel}

\usepackage{tikz}
\usepackage{tkz-euclide}
\usetikzlibrary{3d, angles, arrows, arrows.meta, babel, backgrounds, bending, calc, circuits.logic.US, decorations.markings, decorations.pathmorphing, decorations.pathreplacing, decorations.text, decorations, er, fit, graphs, intersections, matrix, mindmap, patterns, plotmarks, positioning, quotes, svg.path, shadows.blur, shapes.misc, tikzmark, trees, shapes, shadows, through, circuits.logic.US, circuits.logic.IEC, circuits.logic.CDH, circuits.ee.IEC}
\usepackage[american, europeanresistors,americanvoltages,emptydiodes,siunitx]{circuitikz}
	\tikzset{circuit declare symbol = ammeter}
	\tikzset{set ammeter graphic ={draw,generic circle IEC, minimum size=5.5mm,info=center:A}}
	\tikzset{circuit declare symbol = voltmeter}
	\tikzset{set voltmeter graphic ={draw,generic circle IEC, minimum size=5.5mm,info=center:V}}

%   TikZ Grid Paper With Node Coordinates
\makeatletter% from https://tex.stackexchange.com/a/39698/121799
\def\grd@save@target#1{%
	\def\grd@target{#1}}
\def\grd@save@start#1{%
	\def\grd@start{#1}}
\tikzset{
	labeled grid/.style={
		to path={%
			\pgfextra{%
				\edef\grd@@target{(\tikztotarget)}%
				\tikz@scan@one@point\grd@save@target\grd@@target\relax
				\edef\grd@@start{(\tikztostart)}%
				\tikz@scan@one@point\grd@save@start\grd@@start\relax
				\draw[minor help lines] (\tikztostart) grid (\tikztotarget);
				\draw[major help lines] (\tikztostart) grid (\tikztotarget);
				\grd@start
				\pgfmathsetmacro{\grd@xa}{\the\pgf@x/1cm}
				\pgfmathsetmacro{\grd@ya}{\the\pgf@y/1cm}
				\grd@target
				\pgfmathsetmacro{\grd@xb}{\the\pgf@x/1cm}
				\pgfmathsetmacro{\grd@yb}{\the\pgf@y/1cm}
				\pgfmathsetmacro{\grd@xc}{\grd@xa + \pgfkeysvalueof{/tikz/grid with coordinates/major step}}
				\pgfmathsetmacro{\grd@yc}{\grd@ya + \pgfkeysvalueof{/tikz/grid with coordinates/major step}}
				%\foreach \x in {\grd@xa,\grd@xc,...,\grd@xb}
				%\node[anchor=north] at (\x,\grd@ya) {\pgfmathprintnumber{\x}};
				%\foreach \y in {\grd@ya,\grd@yc,...,\grd@yb}
				%\node[anchor=east] at (\grd@xa,\y) {\pgfmathprintnumber{\y}};
				\path foreach \x in {\grd@xa,\grd@xc,...,\grd@xb}
				{foreach \y in {\grd@ya,\grd@yc,...,\grd@yb}
					{ (\x,\y) node[grid with coordinates/grid label] {$(\pgfmathprintnumber{\x},\pgfmathprintnumber{\y})$}}};
			}
		}
	},
	minor help lines/.style={
		help lines,
		step=\pgfkeysvalueof{/tikz/grid with coordinates/minor step},
		draw=none
	},
	major help lines/.style={
		help lines,
		line width=\pgfkeysvalueof{/tikz/grid with coordinates/major line width},
		step=\pgfkeysvalueof{/tikz/grid with coordinates/major step},
		draw=none
	},
	grid with coordinates/.cd,
	minor step/.initial=.2,
	major step/.initial=1,
	major line width/.initial=0pt,
	grid label/.style={below,scale=0.3,color = black!20}
}
\makeatother
 
\begin{document}
%   Transformator 2 - Hochstromtransformtor - Niederspannungstransformator
\begin{circuitikz}[circuit ee IEC, font=\sffamily\footnotesize, scale=1.0]
% Primästromkreis
 \draw 
	(0,1.75) 	
	to [short,o-]																	(0,3)	
	to [ammeter]																	(2,3)	
	to [short,-]																	(3,3)
	to [short,-]																	(3,2.5)
	;
 
% Primärspule mit vielen Windungen
\draw [decorate, decoration={coil,aspect=0.5,amplitude=2.5mm,segment length=1mm}] (3,2.5) -- (3,0.5);
	\draw
	(3,0.5)
	to [short,-]																	(3,0)
	to [short,-]																	(0,0)
	to [short,-o]																	(0,1.25)	
	;
\node at (2.7,2.6) {$n_\text{P}$}
;	

% Beschriftung Strommessgerät im Primästromkreis
\node at (1,3.6) {$I_\text{P}$}
;	
\draw 
    (2,0) 
    to [voltmeter]
    (2,3)
;
 
% Beschriftung Spannungsmessgerät im Primästromkreis
\node at (1.4,1.5) {$U_\text{P}$}
;	

\draw[fill=black] 
    (2,0) 
    circle
    (1.5pt)
;
\draw[fill=black] 
    (2,3) 
    circle
    (1.5pt)
;
\draw
    (5,0) 
    to [voltmeter] 																(5,3)
    ;
\node at (5.6,1.5) {$U_\text{S}$}
;	
\draw[fill=black] 
    (5,0) 
    circle
    (1.5pt)
;
\draw[fill=black] 
    (5,3)
    circle
    (1.5pt)
;
% Sekundärstromkreis
\draw 
(7,0) 	
to [short,-, line width = 0pt]													(4,0)
to [short,-]																	(4,0.5)
;

% Sekundärspule mit wenigen Windungen
\draw [decorate, decoration={coil,aspect=0.5,amplitude=2.5mm,segment length=3mm}] (4,0.5) -- (4,2.5);
\draw
    (4,2.5)
to [short,-]
    (4,3)
to [short]
    (5,3)	
to [ammeter]
    (7,3)
to [short,-]
    (7,3) 
;
\node at (4.3,2.6) {$n_\text{S}$}
;

% veränderbarer Widerstand
\draw
    (7,0)  to [vR]  (7,3)
;

\node at (6,3.6) {$I_\text{S}$}
;	

%% Transformatorkern
% nach links versetzt
\draw [transform canvas={xshift=-1.5pt}]
	(3.5,0.5) -- (3.5,2.5)	
;
% nach rechts versetzt
\draw [transform canvas={xshift=+1.5pt}]
	(3.5,0.5) -- (3.5,2.5)	
;

% AC-Symbol Spannungsquelle
\node at (0,1.5) {$\sim$}
;	
\end{circuitikz}

%anschauliche aber inoffizielle Schaltskizze eines\\ Niederspannungs- bzw.  Hochstromtransformators

\end{document}

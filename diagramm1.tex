\documentclass[tikz, border=10pt]{standalone}
\usepackage[utf8]{inputenc}
\usepackage[T1]{fontenc}

\usepackage{tikz}
	\usetikzlibrary{3d, angles, arrows, arrows.meta, babel, backgrounds, bending, calc, circuits.logic.US, decorations.markings, decorations.pathmorphing, decorations.pathreplacing, decorations.text, decorations, er, fit, graphs, intersections, matrix, mindmap, patterns, plotmarks, positioning, quotes, svg.path, shadows.blur, shapes.misc, tikzmark, trees, shapes, shadows, through, circuits.logic.US, circuits.logic.IEC, circuits.logic.CDH, circuits.ee.IEC}
\usepackage{tkz-euclide}

\usepackage{pgf}
\usepgflibrary{decorations.text}
\usepackage{pgfplots}
\pgfplotsset{compat=newest}

\begin{document}

\tikzset{
	basic/.style  = 	{draw, font=\sffamily, rectangle},
	root/.style   = 	{basic, text width=4cm, rounded corners=2pt, thick, align=center, fill=white!30, minimum height=1cm},
	level 1/.style = 	{basic, text width=4.5cm, rounded corners=2pt, thick, align=center, fill=white!60, minimum height=1cm, sibling distance = 5.5cm}
}

\begin{tikzpicture}
	[
	%edge from parent fork down,
	edge from parent/.style={-latex,draw, very thick},
	level distance=1cm,
	growth parent anchor={south}, 
	nodes={anchor=north}
	]
	
	\node[root] {Arten von Stößen}
	child	{node[level 1] {elastischer Stoß}
	}
	child 	{node[level 1] {teilelastischer Stoß}
	}
	child 	{node[level 1] {inelastischer Stoß}
	};
\end{tikzpicture}
\end{document}

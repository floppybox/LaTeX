\documentclass[border=10pt, tikz]{standalone}
\usepackage[utf8]{inputenc} 
\usepackage[ngerman]{babel}

\usepackage{wasysym}                                        % Electrical and physical symbols \AC \HF \VHF \photon \gluon
\usepackage{siunitx}

\usepackage{tikz}
\usepackage[american, europeanresistors,americanvoltages,emptydiodes,siunitx]{circuitikz}

\usetikzlibrary{circuits.ee.IEC}


\begin{document}

\ctikzset{bipoles/length=0.85cm}
\begin{tikzpicture}[scale=1.25, circuit ee IEC, font=\sffamily\normalsize]
    \draw (0,4) -- (2,4);
    \draw (0,4) -- (0,3.25);
    \draw[fill = white] (0,3.25) circle (2pt);
    \draw[fill = blue] (2,4) circle (1pt);
    
    \draw (0,2) -- (2,2);
    \draw (0,2) -- (0,2.75);
    \draw[fill = white] (0,2.75) circle (2pt);
    \draw[fill = black] (2,2) circle (1pt);
    
    \draw (1,1) to ([shift=(-90:2mm)]1,2) arc (-90:90:2mm) to (1,3);    % Bogen mit Zentrum (1,2)
    
    \draw (1,1) -- (4,1);
    \draw (3,3) -- (4,3);
    
    \draw (4,3) to [resistor] (4,1);
    
    \draw[fill = black] (3,3) circle (1pt);
    \draw[fill = black] (1,3) circle (1pt);
    
    \draw (1,3) to [D-] (2,4);	
    \draw (2,4) to [D-] (3,3);	
    \draw (1,3) to [D-] (2,2);	
    \draw (2,2) to [D-] (3,3);	
    
    \node at (0.01,2.95) {\Large$\mathbf{\AC}$};

   \node at (1.75,3.25) {$\mathsf{_{D_2}}$};
   \node at (2.35,3.25) {$\mathsf{_{D_1}}$};
   \node at (1.75,2.7) {$\mathsf{_{D_3}}$};
   \node at (2.35,2.7) {$\mathsf{_{D_4}}$};
\end{tikzpicture}

\begin{tikzpicture}[scale = 1, samples = 100]
    %% Größe des KOSYs %%%%%%%%%%%%%%%%%%%%%%%%%%%%%%%%%%%%%%%%%%
    
    \pgfmathsetmacro{\xmin}{0}
    \pgfmathsetmacro{\xmax}{15}
    \pgfmathsetmacro{\ymin}{-1}
    \pgfmathsetmacro{\ymax}{1}
    
    %% Definieren der Variablen für Gitter und KOSY %%%%%%%%%%%%%
    
    \pgfmathsetmacro{\xmingrid}{\xmin-1.5}
    \pgfmathsetmacro{\xmaxgrid}{\xmax+2.5}
    \pgfmathsetmacro{\ymingrid}{\ymin-1}
    \pgfmathsetmacro{\ymaxgrid}{\ymax+1.5}
    
    \pgfmathsetmacro{\xminkosy}{\xmin-0.5}
    \pgfmathsetmacro{\xmaxkosy}{\xmax+0.5}
    \pgfmathsetmacro{\yminkosy}{\ymin-0.5}
    \pgfmathsetmacro{\ymaxkosy}{\ymax+0.5}
    
    % Zeichnen des Gitters als Hilfslinien im Hintergrund
    
    \tikzstyle{dotted}= [dash pattern=on 0.1mm off 0.5mm]
    \draw[help lines,line width=0.15mm,step=0.5,style=dotted,black!80,line cap=round] (\xmingrid,\ymingrid) grid (\xmaxgrid,\ymaxgrid);
    
    % Koordinatenachsen mit Beschriftung
    
    \draw[-latex,line width = 0.2mm] (\xminkosy,0) --(\xmaxkosy,0) node[right]{$t$ in \si{\second}};   
    \draw[-latex,line width = 0.2mm] (0,\yminkosy) --(0,\ymaxkosy) node[above, yshift = 0.5ex]{$U$ in \si{\volt}};	
    
    % Einzeichnen der Punkte / Funktionsgraphen
    
	\draw[line width=1pt,domain=0*pi:1*pi, line cap=round, color = magenta] plot (\x,{sin(\x r)});
	\draw[line width=1pt,domain=1*pi:2*pi, line cap=round, color = gray, densely dashed] plot (\x,{sin(\x r)});
	\draw[line width=1pt,domain=1*pi:2*pi, line cap=round, color = magenta] plot (\x,{-sin(\x r)});
	\draw[line width=1pt,domain=2*pi:3*pi, line cap=round, color = magenta] plot (\x,{sin(\x r)});
	\draw[line width=1pt,domain=3*pi:4*pi, line cap=round, color = gray, densely dashed] plot (\x,{sin(\x r)});
	\draw[line width=1pt,domain=3*pi:4*pi, line cap=round, color = magenta] plot (\x,{-sin(\x r)});
	\draw[line width=1pt,domain=4*pi:15.25, line cap=round, color = magenta] plot (\x,{sin(\x r)});
\end{tikzpicture}

\end{document}
